%%%%%%%%%%%%%%%%%
% This is an sample CV template created using altacv.cls
% (v1.6.5, 3 Nov 2022) written by LianTze Lim (liantze@gmail.com). Compiles with pdfLaTeX, XeLaTeX and LuaLaTeX.
%
%% It may be distributed and/or modified under the
%% conditions of the LaTeX Project Public License, either version 1.3
%% of this license or (at your option) any later version.
%% The latest version of this license is in
%%    http://www.latex-project.org/lppl.txt
%% and version 1.3 or later is part of all distributions of LaTeX
%% version 2003/12/01 or later.
%%%%%%%%%%%%%%%%

\documentclass[10pt,a4paper,ragged2e,withhyper]{altacv/altacv}

% Change the page layout if you need to
\geometry{left=1.25cm,right=1.25cm,top=0.5cm,bottom=1.5cm,columnsep=1.2cm}

% multi-lang config
\usepackage[english,spanish]{babel}
\newcommand\en[1]{\iflanguage{english}{#1}{}}
\newcommand\es[1]{\iflanguage{spanish}{#1}{}}

% The paracol package lets you typeset columns of text in parallel
\usepackage{paracol}

% Change the font if you want to, depending on whether
% you're using pdflatex or xelatex/lualatex
\ifxetexorluatex
  % If using xelatex or lualatex:
  \setmainfont{Roboto Slab}
  \setsansfont{Lato}
  \renewcommand{\familydefault}{\sfdefault}
\else
  % If using pdflatex:
  \usepackage[rm]{roboto}
  \usepackage[defaultsans]{lato}
  % \usepackage{sourcesanspro}
  \renewcommand{\familydefault}{\sfdefault}
\fi

% Change the colours if you want to
\definecolor{SlateGrey}{HTML}{2E2E2E}
\definecolor{LightGrey}{HTML}{666666}
\definecolor{DarkPastelRed}{HTML}{450808}
\definecolor{PastelRed}{HTML}{8F0D0D}
\definecolor{GoldenEarth}{HTML}{E7D192}
\colorlet{name}{black}
\colorlet{tagline}{PastelRed}
\colorlet{heading}{DarkPastelRed}
\colorlet{headingrule}{GoldenEarth}
\colorlet{subheading}{PastelRed}
\colorlet{accent}{PastelRed}
\colorlet{emphasis}{SlateGrey}
\colorlet{body}{LightGrey}

% Change some fonts, if necessary
\renewcommand{\namefont}{\Huge\rmfamily\bfseries}
\renewcommand{\personalinfofont}{\footnotesize}
\renewcommand{\cvsectionfont}{\LARGE\rmfamily\bfseries}
\renewcommand{\cvsubsectionfont}{\large\bfseries}


% Change the bullets for itemize and rating marker
% for \cvskill if you want to
\renewcommand{\itemmarker}{{\small\textbullet}}
\renewcommand{\ratingmarker}{\faCircle}

\begin{document}
\selectlanguage{\jobname}
\begin{flushright}
{\scriptsize {
  \github{pablouser1/My-CV}
  \linebreak
  \en{Generated: \today}
  \es{Generado el \today}
}}
\end{flushright}
\medskip
\name{Pablo Ferreiro}
\tagline{
  \en{Fullstack Dev}
  \es{Desarrollador Fullstack}
}
%% You can add multiple photos on the left or right
\photoR{2.8cm}{me}
% \photoL{2.5cm}{Yacht_High,Suitcase_High}

\personalinfo{%
  \email{me@pabloferreiro.es}
  \location{Benalmádena, Málaga, Spain}
  \homepage{pabloferreiro.es}
  \github{pablouser1}
  \linkedin{pablo-ferreiro-romero-bba809257}
  \twitter{pablouser1}
  \NewInfoField{telegram}{\faTelegram}[https://t.me/]
  \telegram{pablouser1}
  \printinfo{\faHashtag}{@pablouser1:matrix.org}[https://matrix.to/\#/@pablouser1:matrix.org]
}

\makecvheader

%% Set the left/right column width ratio to 6:4.
\columnratio{0.6}

% Start a 2-column paracol. Both the left and right columns will automatically
% break across pages if things get too long.
\begin{paracol}{2}
\cvsection{
  \en{About me}
  \es{Acerca de}
}

\en{Hi there! I'm a passionate open source developer.}
\es{¡Hola! Soy un desarrollador open-source apasionado por el mundo de la programación.}
\linebreak
\linebreak
\en{I started getting into programming when I got my first ever laptop, a Windows XP with 512MB RAM. Since then, I started making programs to make my life easier and started sharing them with others who may find it useful.}
\es{Empecé a programar cuando conseguí mi primer portátil, con Windows XP y 512MB de RAM. Desde entonces, desarrollé un gran número de programas para hacer mi vida más fácil y empecé a compartirlos con otras personas.}

\cvsection{
  \en{Experience}
  \es{Experiencia}
}

\cvevent{Webdev}{Freelancer}{
  \en{July 2022 -- Ongoing}
  \es{Julio 2022 -- Actualidad}
}{
  \en{Benalmádena, Spain (Remote)}
  \es{Benalmádena, España (Remoto)}
}
\begin{itemize}
\item {
  \en{Blog made from scratch using PHP7}
  \es{Blog hecho desde cero usando PHP, sin frameworks}
}
\item Web: \homepage{iescerrodelviento.es}
\end{itemize}

% -- PROJECTS -- %
\cvsection{
  \en{Projects}
  \es{Proyectos}
}

\cvevent{ProxiTok}{\github{pablouser1/ProxiTok}}{
  \en{Jan 2022 -- Ongoing}
  \es{Enero 2022 -- Actualidad}
}{}
\begin{itemize}
    \item {
      \en{Open source alternative frontend for TikTok made using PHP}
      \es{Alternativa open-source para TikTok hecho con PHP}
    }
    \item {
      \en{Over 1.5K stars in Github and \textasciitilde{}167K requests per-day on the official instance}
      \es{Tiene cerca de 1.5K estrellas in Github y ~167K solicitudes por día en la instancia oficial}
    }
\end{itemize}

\divider

\cvevent{plugin.video.filmin}{\github{pablouser1/plugin.video.filmin}}{
  \en{Jul 2021 -- Ongoing}
  \es{Julio 2021 -- Actualidad}
}{}
\en{(\textit{unofficial}) Open source plugin of Filmin for Kodi Made using Python}
\es{Plugin (\textit{no oficial}) de código abierto para usar Filmin en Kodi hecho en Python}
% -- END PROJECTS -- %

\cvsection{
  \en{Volunteering}
  \es{Voluntariado}
}
\cvevent{Hackers Week 9}{\homepage{hackersweek.es}}{
  \en{20-24th of February 2023}
  \es{20-24 de Febrero 2023}
}{ETS Ingeniería Informática}

\divider

\cvevent{UAD360 3}{\homepage{uad360.es}}{
  \en{16-17th of June 2023}
  \es{16 y 17 de Junio 2023}
}{ETS Ingeniería Informática}

\medskip

\newpage
\switchcolumn

\cvsection{
  \en{My Life Philosophy}
  \es{Mi filosofía de vida}
}

\begin{quote}
``Work smarter, not harder.''
\end{quote}

\cvsection{
  \en{Most Proud of}
  \es{Orgulloso de}
}

\cvachievement{\faUsers}{
  \en{building an OSS community}
  \es{construir una comunidad de OSS}
}{
  \en{Having people willing to expend time/money on a project you've started}
  \es{Me encanta saber que hay gente dispuesta a invertir tiempo/dinero en un proyecto que yo empecé}
}

\divider

\cvachievement{\faServer}{
  \en{being (almost) fully self-hosted}
  \es{tener (casi) todo self-hosted}
}{
  \en{Cloud, email, sync... is all moved to my own VPS}
  \es{La nube, email, sync... está todo alojado en mi VPS personal}
}

\cvsection{
  \en{Strengths}
  \es{Fortalezas}
}

\cvtag{
  \en{Quick learner}
  \es{Aprendizaje rápido}
}
\cvtag{
  \en{Hard-working}
  \es{Trabajador}
}

\divider\smallskip

\cvtag{PHP}
\cvtag{Python}
\cvtag{GoLang}
\cvtag{Typescript}

\divider\smallskip

\cvtag{Laravel}
\cvtag{Flutter}
\cvtag{Vue}\\
\cvtag{SolidJS}
\cvtag{Svelte}

\divider\smallskip

\cvtag{Linux SysAdmin}

\cvsection{
  \en{Languages}
  \es{Idiomas}
}

\cvskill{
  \en{Spanish (Native)}
  \es{Español (Nativo)}
}
{5}
\divider

\cvskill{{
  \en{English}
  \es{Inglés}
}(B2)}{3.5}
\divider

\cvskill{{
  \en{French}
  \es{Francés}
}(A2)}{1.5}
\medskip

\cvsection{
  \en{Education}
  \es{Educación}
}

\cvevent{Bachillerato \en{(High school)}}{IES Al-Baytar}{2019 -- 2020}{}
\begin{itemize}
    \es{\item Rama tecnológica}
    \item Bilingual modality
\end{itemize}

\divider

\cvevent{
  \en{Computer Engineering}
  \es{Ingeniería de Computadores}
}{Universidad de Málaga}{2021 -- {
  \en{Ongoing}
  \es{Actualidad}
}}{}
\begin{itemize}
    \item {
      \en{Currently on the 2nd year}
      \es{Actualmente en el segundo año}
    }
    \item {
      \en{Bilingual modality}
      \es{Modalidad bilingüe}
    }
\end{itemize}

\end{paracol}

\end{document}
